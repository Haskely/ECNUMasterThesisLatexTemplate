% !TeX root = ./main.tex

\documentclass[zihao = -4,AutoFakeBold]{ctexbook}
% \documentclass[fontset = ubuntu]{ctexbook} % 跟据文档,ctex包应该能够自动检测操作系统,并且overleaf服务器系统为linux。然而overleaf还是提示了字体不存在警告:Package fontspec Warning: Font "FandolSong-Regular" does not contain requested Script "CJK". 因此改为 ubuntu
\ctexset{
    bibname = {参考文献}, % 使用的是 biblatex 库,也可以在\printbibliography[heading=bibintoc,title={参考文献}] 中指定名称,后者设置会覆盖前者
    tocdepth = {section}, % 设置对章节标题编入目录的层次数。tocdepth 的值可以是一个整数,也可以是 part, chapter 等名称,具体需要见文档
    chapter={
        % name={第,章},
        % number=\arbic{chapter}, % 编号使用阿拉伯数字
        format={\heiti \zihao{3} \bfseries},% 一级标题用3号粗黑体
        beforeskip={-10pt},
        afterskip={20pt},
        pagestyle = fancy % 设置chapter首页使用fancy格式
    },
    section={
        format={\raggedright \heiti \zihao{4} \bfseries}, % 二级标题用4号粗黑体
        afterskip={1ex plus 0.2ex}
    },
    subsection={
        format={\raggedright \heiti \zihao{-4} \bfseries}, % 三级标题用小4号粗黑体
        afterskip={0.5ex plus 0.1ex}
    },
} 

% 设置正文字体
% 西文字体
% \usepackage{newtxmath} % https://liam.page/2017/01/10/Times-New-Roman-and-LaTeX/ 
% mathptmx 是一个为 LaTeX 设计的字体宏包,它会将默认 rmfamily 设置为 Nimbus Roman No9 L;而将数学字体设置为对应的 Italic 字形(不足的部分使用了 CM/RSFS/Adobe Symbol 等字体)。这是同时修改默认文本字形和默认数学字形为 Times 字形最干净的宏包。
\setmainfont{Times New Roman}
% \setsansfont{Times New Roman}
% \setmonofont{Times New Roman}
% 中文字体 默认即为宋体 所以不需要修改
% \setCJKmainfont{Source Han Serif SC}
% \setCJKsansfont{Source Han Sans SC}
% \setCJKmonofont{Source Han Sans SC}
% \setCJKfamilyfont{boldsong}{Source Han Serif SC Heavy}

% 设置页面格式,参考 geometry 的用法
\usepackage{geometry} %
\geometry{
    a4paper,
    left=3.18cm,
    right=3.18cm,
    top=2.54cm,
    bottom=2.54cm,
    headheight=0.55cm,
    % headsep=1cm,
    % footskip=15mm
} %

% 设置页眉页脚
\usepackage{fancyhdr} %
\pagestyle{fancy} % 使用 fancy 风格
\fancyhf{} % 清除所有默认的页眉页脚
\fancyhead[C]{华东师范大学硕士学位论文} % 页眉应写明“华东师范大学博(硕)士学位论文”字样
\fancyfoot[C]{\thepage} % 页脚居中页码
% 学位论文的页码,前置部分采用罗马数字单独编连续码,正文和结尾部分用阿拉伯数字编连续码,这种设置实际上已经被ctexbook以及\frontmatter \mainmatter 涵盖了
\makeatletter 
\def\cleardoublepage{
    \clearpage
    \if@twoside 
        \ifodd
            \c@page
        \else
            \begingroup
                \mbox{}
                \vspace*{\fill}
                \begin{center}
                    本页留白
                \end{center}
                \vspace{\fill}
                \thispagestyle{empty}
                \newpage
                \if@twocolumn
                    \mbox{}
                    \newpage
                \fi
            \endgroup
        \fi
    \fi
}
\makeatother % fancyhdr 文档中 20 Those blank pages 提到:In the book class when the openany option is not given or in the report class when the openright option is given, chapters start at odd-numbered pages, half of the time causing a blank page to be inserted. Some people prefer this page to be completely empty, i.e. without headers and footers. This cannot be done with \thispagestyle as this command would have to be issued on the previous page. ...... As the \pagestyle{empty} is enclosed in a group it only affects the page that may be generated by the \cleardoublepage. You can of course put the above in a private command. If you want to have this done automatically at each chapter start or when you want some other text on the page then you must redefine the \cleardoublepage command. 在没有openany选项的书籍类中,或在openright选项的报告类中,章节从奇数页开始,有一半的时间会导致插入一个空白页。有些人喜欢这个页面是完全空的,即没有页眉和页脚。这不能用\thispagestyle来做,因为这个命令必须在前一页发出。然而,要做到这一点,并没有什么必要的魔法。当然,你可以把上述内容放在一个私人命令中。如果你想在每一章开始时或当你想在页面上有一些其他的文字时自动完成,那么你必须重新定义 \cleardoublepage 命令。随后给出了参考样例。这样可以避免书籍的偶数空白页存在页眉页脚。

\usepackage[hidelinks]{hyperref} % 使目录与引用可跳转

\usepackage{graphicx} % 用来插入图片 \includegraphics
\usepackage{subcaption} % 用来一个图由两个或两个以上分图组成

\usepackage{tabularx} % 用来生成可变长度表格
\usepackage{booktabs} % 用来生成三线表格
% \usepackage{threeparttable} % 表格如果有附注,尤其是需要在表格中进行标注时,可以使用 {threeparttable} 宏包
% \usepackage{longtable} % 如某个表需要转页接排,可以使用 {longtable} 宏包,需要在随后的各页上重复表的编号。
% \usepackage{multirow} % 用来设计跨行表格单元

% 算法环境可以使用 {algorithms} 或者 {algorithm2e} 宏包
% \usepackage{algorithm}
% \usepackage{algorithmic}
\usepackage[linesnumbered,ruled,vlined]{algorithm2e} 
\renewcommand{\algorithmcfname}{算法}
\SetKwInOut{KwIn}{输入}
\SetKwInOut{KwOut}{输出}

\usepackage{amsmath} % 数学公式包
% \allowdisplaybreaks[4]
% \usepackage{upgreek} % 更多大写希腊字母
\usepackage{amssymb} % 特殊字体
% \usepackage{siunitx} % 一些单位,数字格式
 

\usepackage{amsthm} % 数学定理包
\newtheorem{theorem}{定理}[chapter]
\newtheorem{lemma}[theorem]{引理}
\newtheorem{corollary}[theorem]{推论}
\newtheorem{proposition}[theorem]{命题}
\newtheorem{property}[theorem]{性质}
\newtheorem{definition}{定义}[chapter]
\newtheorem{remark}{注记}[chapter]

\usepackage[backend=biber,style=gb7714-2015]{biblatex} %Reference
\addbibresource{reference.bib}

\usepackage{float} % 为了解决 `!h' float specifier changed to `!ht'.