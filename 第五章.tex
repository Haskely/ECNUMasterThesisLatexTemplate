% !TeX root = ./main.tex

\chapter{总结与展望}

本文基于 PINN 的思路提出了 Navier-Stokes/Darcy 耦合模型的神经网络求解算法 cPINNs。文章针对两个物理区域提出使用双通道的神经网络结构进行数值拟合,并利用交界面条件构造的损失函数使两个网络能够协同训练,达到求解的目的,同时利用万能表示定理一定程度上证明了损失函数构造的合理性,同时本文通过若干数值实验的测试,在一定程度上探索并验证了深度学习技术在耦合物理模型偏微分方程正反问题领域的求解能力。

在数值实验中文章还针对不同的网络层数与激活函数进行了简单探索,平衡了网络参数量与准确性的关系。本文提出的算法针对耦合模型的稳态形式以及不同的交界面条件,只需按第二章所描述同样的方法重新构造损失函数便可进行求解,说明算法具有一定的可扩展性。针对初始条件复杂的区域反问题和物理量反问题,算法只要对涉及到已知物理量真解的损失函数部分进行替换就可进行迁移;对于方程自身存在未知参数的反问题,将未知参数等同于网络自身参数的一部分即可求解。实验效果表明算法在这些反问题的解决上也拥有较高的潜力。

受时间与技术所限,文章测试的网络结构配置数量有限,也没有对更多的激活函数或网络结构如卷积神经网络(CNN),循环神经网络(RNN)等进行深入探索,此外训练算法的设置更多来源于实验中的经验,而严谨地对不同训练算法对网络拟合效果影响进行探究也是以后可以关注的方向之一。另一方面,相信本文算法对于更多更复杂的区域或方程模型数值求解问题上也会拥有潜力,值得未来进行探索。